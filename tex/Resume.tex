%!TEX root = ../thesis.tex
\begin{resume}
% \chapter{个人简历、在读期间发表的学术论文与研究成果}
	\resumeitem{个人简历}

	林晓东,男,1990年3月生。

	2013年6月毕业于同济大学,土木工程专业,获学士学位。

	2013年9月入同济大学,隧道及地下建筑工程专业,攻读博士研究生。

	\resumeitem{已发表论文}

	\hangindent 1.5em
	[1]~Xiaojun Li, Xiaodong Lin, Hehua Zhu, Xiuzhi Wang, and Zhaoming Liu. "Condition assessment of shield tunnel using a new indicator: The tunnel serviceability index." Tunnelling and Underground Space Technology 67 (2017): 98-106.

	\hangindent 1.5em
	[2] Lin, Xiaodong, Xiaojun Li, and Hehua Zhu. "A Web-based Digital Platform for Shield Tunnel Operation Management." In Information Technology in Geo-Engineering: Proceedings of the 2nd International Conference (ICITG) Durham, UK, vol. 3, p. 130. IOS Press, 2014.

	\hangindent 1.5em
	[3] Li, Xiaojun, Xiaodong Lin, Hehua Zhu, Xiuzhi Wang, and Zhaoming Liu. "A BIM/GIS-based management and analysis system for shield tunnel in operation.

	\hangindent 1.5em
	[4]~朱合华, 李晓军, 林晓东. 基础设施智慧服务系统 (iS3) 及其应用. 土木工程学报 51.1 (2018): 1-12.

	\hangindent 1.5em
	[5]~林晓东, 李晓军, 林浩. 集成GIS/BIM的盾构隧道全寿命管理系统研究. 隧道建设, 38(6), 963-970.
	
	% \resumeitem{待发表论文}

	% \hangindent 1.5em
	% [1]~丁文其, 林晓东, 李晓军. 基于GIS的地铁盾构隧道状态网格化评估

	% \hangindent 1.5em
	% [1]~李晓军, 林晓东, 王建, 史海欧, 袁泉, 翟利华. 考虑空间关联性的轨道交通盾构隧道沉降时间序列模型

	% \hangindent 1.5em
	% [2]~李晓军, 林晓东, 朱合华, 陈楠. 考虑发展趋势和动态权重的盾构隧道健康状态模糊层次分析法.

	\resumeitem{软件著作权}

	\hangindent 1.5em
	[1]~李晓军, 林晓东, 智慧基础设施(iS3)基础平台软件, 2015SR162360

	\hangindent 1.5em
	[2]~李晓军, 林晓东, 盾构隧道结构建养一体数字化管理软件, 2015SR162185

	\resumeitem{参与科研项目}

	\hangindent 1.5em
	[1]~参与国家重点基础研究发展计划(973 计划),“城市轨道交通地下结构性能演化与感控基础理论(2011CB013800)”,负责课题六中盾构隧道服役性能评估与预测分析的研究。

	\hangindent 1.5em
	[2]~参与国家自然科学基金,“拼装式盾构结构服役性能退化模型及在隧道维养应用中研究(51478341)”,负责盾构隧道结构服役性能退化模型的维养应用方法研究。

\end{resume}